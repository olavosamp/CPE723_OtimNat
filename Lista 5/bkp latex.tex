\documentclass{article}

\input{packages}

\begin{document}

\begin{flushleft}
	\large{Lista 5}
\end{flushleft}

\begin{flushright}
{CPE723 - Otimização Natural} \\
{Olavo Sampaio}\\
   \vspace{1cm}
\end{flushright}

% Questão 1
\section{}

\begin{itemize}
	\item[] Representação: Trinta dimensões de estados, com um sigma para cada, igual ao ES com mutação descorrelacionada e passos diferentes. Assim, o vetor de estados tem dimensão 1 x 60. Os valores foram inicializados de forma aleatória uniforme no intervalo aberto (-30, 30).
	\item[] População: 30
	\item[] Seleção de pais: Todos os indivíduos
	\item[] Mutação: Descorrelacionada, com um passo por dimensão. Atualização do parâmetros segue as equações do ES.
	\item[] Seleção de Sobreviventes: Torneio, com q=10.

\end{itemize}

\paragraph{}Após 100 rodadas, as estatísticas dos melhores valores são:
\begin{itemize}
	\item[] Média: \hspace{1.1cm} 1,37
	\item[] Desvio-Padrão:	0,737
\end{itemize}

\paragraph{}O mínimo global da função de Ackley é 0, então o algoritmo convergiu para um mínimo local próximo.
Na Fig. 1 pode-se observar a evolução de uma população com esse algoritmo.
Usando o método de seleção dos 30 melhores, o algoritmo converge para o mínimo global rapidamente, como pode-se observar na Fig. 2.

% Questão 2
\section{}

\paragraph{}\large{Considerar a diferença entre dois métodos de avaliação da velocidade de um Algoritmo Evolucionário: número de geração e número de avaliações da função custo.}
\paragraph{}O número de gerações é uma medida menos confiável que o número de avaliações da função custo. Um algoritmo pode ter um número variável de avaliações de função por geração, o que faz com que o custo computacional de cada geração seja diferente. Nesse caso, dois algoritmos que executem um mesmo número de gerações podem ter custo computacional diferente. Portanto, medir a função custo é uma forma mais segura de avaliação.

% Questão 3
\section{}

\paragraph{}\large{Argumente porque a taxa de mutação deve ser aumentada ou diminuída durante a execução de um algoritmo (ao longo das gerações).}
\paragraph{}No início do algoritmo, a taxa de mutação pode ser elevada, para que ele percorra rapidamente o espaço de soluções em busca dos ótimos. À medida que ele se aproxima de um ótimo, é vantajoso que a taxa seja reduzida, de modo que ele possa encontrar um ponto cada vez mais próximo do ótimo. Caso ele continue com uma taxa elevada, o algoritmo não converge para um valor tão próximo do ótimo, por não conseguir realizar um ajuste fino da solução no seu entorno.

\paragraph{}Essa estratégia de adaptação está representada, por exemplo, na \textit{Regra de 1/5 de Sucesso} de Rechenberg, que propõe que o tamanho da pertubação aplicada seja alterada de acordo com a taxa de sucesso da mutação. Ela é definida como a proporção de filhos que são mais aptos do que seus pais após a perturbação. Caso a taxa de sucesso esteja elevada, deve-se aumentar o tamanho das perturbações futuras, caso contrário, ele deve ser reduzido.

% Ques
\section{}



\end{document}
